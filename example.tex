\documentclass[aps,prl,reprint,twocolumn,superscriptaddress,floatfix,letterpaper,longbibliography]{revtex4-1}
%-----------------------------------------------------------------
\usepackage{amssymb,amsfonts,amsmath}
\usepackage{CJK} % for Chinese characters

% to stick the position of figures (\begin{figure}[!htbp]) 
\usepackage{float} 

% add hyperlinks to labels
\usepackage{hyperref}
\hypersetup{colorlinks=true}
\usepackage[all]{hypcap} % let hyperlinks correctly point to figures rather than their captions; must be loaded after the hyperref package!

% change the font to Times Roman from the incredibly ugly Computer Modern
\usepackage{times}

% use the "hooked" Greek symbol \uptau
\usepackage{upgreek}

% Include figure files
\usepackage{graphicx}

\usepackage{color}

\DeclareMathOperator{\arccosh}{arcosh}
\DeclareMathOperator{\cotha}{coth}
\DeclareMathOperator{\tanha}{tanh}
\DeclareMathOperator{\re}{Re}
\DeclareMathOperator{\im}{Im}
\DeclareMathOperator{\res}{Res}

\usepackage{filecontents}
\begin{filecontents}{\jobname.bib}
	@article{Ref1,
		author={Y.-L. L. Fang},
		title={This is the first reference},
		journal={Journal of blah},
		volume={1},
		pages={1},
		year={2011}
	}
	@article{Ref2,
		author={Y.-L. L. Fang},
		title={This is the second reference},
		journal={Journal of blah},
		volume={1},
		pages={2},
		year={2012}
	}
	@article{Ref3,
		author={Y.-L. L. Fang},
		title={This is the third reference},
		journal={Journal of blah},
		volume={1},
		pages={3},
		year={2013}
	}
	@article{Ref4,
		author={Y.-L. L. Fang},
		title={This is the fourth reference},
		journal={Journal of blah},
		volume={1},
		pages={4},
		year={2014}
	}
	@article{Ref5,
		author={Y.-L. L. Fang},
		title={This is the fifth reference},
		journal={Journal of blah},
		volume={1},
		pages={5},
		year={2015}
	}
\end{filecontents}

% seperate bibliography for main text and supplementary material
\usepackage{bibunits}

% ----------------------------------------------------------------------------
% main text starts

\begin{document}
\begin{bibunit}[apsrev4-1]
\title{This is a minimal working example for using bibunits with REVTeX 4.1}
\author{Leo Fang}
\affiliation{Department of Physics, Duke University, P.O. Box 90305, Durham, North Carolina 27708-0305, USA}
\date{\today}

\begin{abstract}
We investigate this and that, and we find this and that, which we think worth publishing on PRL.
\end{abstract}
\maketitle

This is all about blahblahblah \cite{Ref1, Ref2, Ref3}. In this work, we find that 
\begin{equation}
   E = mc^2, \label{eq1}
\end{equation}
and Eq.~\eqref{eq1} is our main result.

\emph{Acknowledgments.} Thank you very much.

\nocite{apsrev41Control}
\putbib[\jobname,REVTeX-longbib]
\end{bibunit}

\widetext
\clearpage

% ----------------------------------------------------------------------------
% supplementary material starts
\begin{bibunit}[apsrev4-1]

\renewcommand{\bibnumfmt}[1]{[S#1]}
\renewcommand{\citenumfont}[1]{S#1}
\renewcommand{\theequation}{S\arabic{equation}}
\renewcommand{\thefigure}{S\arabic{figure}}
\renewcommand{\thepage}{S\arabic{page}}  
\renewcommand{\thesection}{S\arabic{section}}   
\renewcommand{\thetable}{S\arabic{table}}
\setcounter{equation}{0}
\setcounter{figure}{0}
\setcounter{page}{1}
\setcounter{section}{0}

\begin{center}
	\textbf{\large Supplemental Material for ``This is a minimal working example for using bibunits with REVTeX 4.1''}\\
	\vspace{15pt}
	Leo Fang\\
	(Dated: \today)
\end{center}

I am citing other references not cited in the main text \cite{Ref4, Ref5}. Now cite the papers already appeared in the main text, but in different order \cite{Ref2, Ref1, Ref3}. 

Now I present a new equation
   \begin{equation}
      F=ma \label{Seq1}
   \end{equation}
and then I refer to it \eqref{Seq1}. Now I refer the equation in the main text \eqref{eq1}.

\nocite{apsrev41Control}
\putbib[\jobname,REVTeX-longbib]
\end{bibunit}

\end{document}
